\section{Preliminary Notions}
\subsection{Set Theory}
\begin{enumerate}
\item[9.] Recognizing the set subtraction $A-B$ as equivalent to $A \cap B^c$,
we have (a)-(f) as a consequence of De Morgan's Laws. We demonstrate one
example. For example we deduce that $X \cup (Y - Z) = X \cup Y - X \cup Z$,
using which we can prove as follows (b): $A \cdot (B + C) = A \cup ((B-C) \cap
(C-B)) = (A \cup (B-C)) \cap (A \cup (C-B)) = (A \cup B-A \cup C) \cap (A \cup
C - A \cup B) = $

\item[12.] That the relation defined is an equivalence relation follows
from the facts: a) $x \sim x$ because $n | 0$, b) $x \sim y \implies y
\sim x$ because $n | (x-y) \implies n | (y-x)$, and c) $x \sim y, y \sim
z \implies x \sim z$ because $n | (x-y) + (y-z) = (z-x)$. No two
distinct equivalence classes containing each of $0, 1, \dots, n-1$ are
identical since that would imply $n | (i - j), 0 \leq i < j < n$, which
is a contradiction.  Hence there are at least $n$ equivalence classes
$cl(0), cl(1), \dots, cl(n-1)$. Now, by the Euclidean division
algorithm, any integer $k \in S$, can be divided by $n$ to yield a
remainder $r$ that satisfies $k = qn + r$, where $0 \leq r < n$. So, $k
\sim r$ as $n | (k-r) = qn$, and hence belongs to $cl(r)$.
\end{enumerate}

\subsection{Mappings}
\begin{enumerate}
\item[6.]
\item[10.] We form a one-to-one correspondence $f:Q \to N \times N$.
\item[11.]
\item[12.]
\item[13.]
\item[14.]
\item[15.]
\item[16.]
\end{enumerate}

\subsection{}
\begin{enumerate}
\item[]
\end{enumerate}
